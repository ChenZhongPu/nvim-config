\documentclass{article}
\usepackage{geometry}
\geometry{
	a4paper,
	total={170mm,257mm},
	left=20mm,
	top=20mm,
}
\usepackage{graphicx}
\usepackage{titling}
\usepackage{xcolor}
\usepackage{amsmath}
\usepackage{amsthm}
\newtheorem{theorem}{Theorem}

\usepackage{hyperref}
\hypersetup{
	colorlinks=true,
	linkcolor=blue,
	filecolor=magenta,
	urlcolor=cyan,
}
\urlstyle{same}
\title{Final Project}
\author{CHEN Zhongpu}
\date{Fall \the\year}

\usepackage{fancyhdr}
\fancypagestyle{plain}{%  the preset of fancyhdr 
	\fancyhf{} % clear all header and footer fields
	\fancyfoot[L]{SWUFE}
	\fancyhead[L]{Data Structures}
	\fancyhead[R]{\theauthor}
}
\makeatletter
\def\@maketitle{%
	\newpage
	\null
	\vskip 1em%
	\begin{center}%
		\let \footnote \thanks
		{\LARGE \@title \par}%
		\vskip 1em%
			{\large \@date}%
	\end{center}%
	\par
	\vskip 1em}
\makeatother

\usepackage{lipsum}

\begin{document}

\maketitle

\section*{Notice}
\begin{itemize}
	\item You should finish them independently, and plagiarism is strictly prohibited.
	\item You can choose \textbf{ONE} of the following projects.
	\item Please submit your work as a PDF file.
	\item Please upload your code in GitHub, and the link is included in the PDF.
	\item Python 3.7+ or JDK 16+.
	\item 37 marks in total.
\end{itemize}

\section{Project One}

A \href{https://en.wikipedia.org/wiki/K-d_tree}{k-d tree} (short for k-dimensional tree) is a space-partitioning data structure for organizing points in a k-dimensional space.

Please complete the code based on \href{https://github.com/ChenZhongPu/data-structure-swufe/blob/master/code/python/tree/kd_tree.py}{kd\_tree.py} or \href{https://github.com/ChenZhongPu/data-structure-swufe/blob/master/code/java/tree/src/main/java/org/swufe/KDTree.java}{KDTree.java}. In this project, we only consider points in two dimensions. Note that you cannot use any third-party library.

\begin{enumerate}
	\item Explain the existing code (5 marks).
	\item Implement and explain \texttt{insert()} and \texttt{range()} (English writing style will be judged, 22 marks).
	\item Analyze the time complexity of range query (5 marks).
	\item Visualize the time performance between k-d tree method and naive method (5 marks).
	\item (Bonus) Implement the nearest neighbor query (5 marks).
\end{enumerate}

Also note that if you are not able to explain your code in \textcolor{red}{Q2}, then you will be risking of losing all marks in \textcolor{red}{Q2}.

\section{Priority queues}
We have learned how to represent a heap with an array. In this project, we are going to implement it with a singly linked list, where each node is defined as:

\begin{verbatim}
class Node:
    def __init__(self, key=None, next=None):
        self.key = key
        self.next = next
\end{verbatim}

\lipsum
\end{document}
